\documentclass[12pt, a4paper]{article}

\usepackage{amsmath}
\usepackage{amssymb,dsfont}
\usepackage[margin=1in]{geometry}
\usepackage{mathtools}
\usepackage[parfill]{parskip}
\usepackage{siunitx}
\usepackage{lipsum}
\usepackage{microtype}
\usepackage{booktabs}
\usepackage{hyperref}
\usepackage{bm}
\usepackage{xcolor}
\usepackage{cite}
\usepackage{graphicx}
\usepackage{float}
\usepackage{longtable}
\usepackage{multirow}
\usepackage[section]{placeins}

\newcommand{\pvec}[1]{\vec{#1}\mkern2mu\vphantom{#1}}

\begin{document}
\title{Algorithmic Search for Type-I Topological Magnons}
\maketitle

\section{Ingredients}
\section{Group-subgroup relations}
Let $G$ and $G'\subset G$ be two magnetic space groups. In their corresponding standard settings, denote the conventional lattice vectors and the choice of origin by
\begin{equation}
  \{\bm{a}, \bm{b},\bm{c}; \bm{o}\}
\end{equation} 
and
\begin{equation}
  \{\bm{a}', \bm{b}',\bm{c}'; \bm{o}'\},
\end{equation} 
respectively. Additionally, define the 4\texttimes4 matrix
\begin{equation}\label{sfMblocks}
  \mathsf{M}=
  \left(
  \begin{array}{@{}c|c@{}}
    \begin{matrix}
      \phantom{0}&& \\
      &M& \\
      \phantom{0}&& 
    \end{matrix}&\bm{\delta}\\
    \hline
    \begin{matrix}
      0&0&0
    \end{matrix}&1
  \end{array}
  \right)
\end{equation} 
that maps the standard setting of the group to that of the subgroup. More precisely, $\mathsf{M}$ is defined by the relation
\begin{equation}\label{subgroupToGroupTranslations}
  \left(
  \begin{array}{@{}ccc|c@{}}
    \bm{a}'&\bm{b}'&\bm{c}' &\bm{o}' \\
    \hline
      0&0&0&1
  \end{array}
  \right)
  =
  \left(
  \begin{array}{@{}ccc|c@{}}
      \bm{a}&\bm{b}&\bm{c}&\bm{o}\\
    \hline
      0&0&0&1
  \end{array}
  \right)
  \mathsf{M}.
\end{equation} 
Therefore, for a general point expressed as
\begin{align}
  \bm{r} &= r_1\bm{a}+r_2\bm{b}+r_3\bm{c}+\bm{o}\\
         &= r_1'\bm{a}'+r_2'\bm{b}'+r_3'\bm{c}'+\bm{o}',
\end{align} 
the mapping between the group and subgroup standard coordinates reads
\begin{equation}
  \begin{pmatrix}
    r_1\\
    r_2\\
    r_3\\
    1
  \end{pmatrix}
 =
  \mathsf{M}
  \begin{pmatrix}
    r'_1\\
    r'_2\\
    r'_3\\
    1
  \end{pmatrix},
\end{equation} 
or equivalently,
\begin{equation}
  \begin{pmatrix}
    r_1\\
    r_2\\
    r_3
  \end{pmatrix}
  =
  M
  \begin{pmatrix}
    r'_1\\
    r'_2\\
    r'_3
  \end{pmatrix}
  +
  \begin{pmatrix}
    \delta_1\\
    \delta_2\\
    \delta_3
  \end{pmatrix}.
\end{equation} 

Since the group (standard) reciprocal translation vectors ($\bm{a}^\star$, $\bm{b}^\star$, $\bm{c}^\star$) are defined by
\begin{equation}
  {\begin{pmatrix}
    \bm{a}^\star& \bm{b}^\star & \bm{c}^\star\\
  \end{pmatrix}}^{T}
  \begin{pmatrix}
    \bm{a}& \bm{b} & \bm{c}\\
  \end{pmatrix}
  =
  \mathds{1}
\end{equation} 
(and similarly for the subgroup vectors), 
the mapping of the reciprocal translation vectors from the group to the subgroup is obtained with
\begin{equation}
  \begin{pmatrix}
    \bm{a}^{\star\prime}& \bm{b}^{\star\prime}& \bm{c}^{\star\prime}\\
  \end{pmatrix}
  =
  \begin{pmatrix}
    \bm{a}^{\star}& \bm{b}^{\star}& \bm{c}^{\star}\\
  \end{pmatrix}
  {M^{-1}}^{T},
\end{equation} 
where $M$ is given in Eq.~\ref{sfMblocks}.

Therefore, for any $k$-point
\begin{align}
  \bm{k}&=k_1\bm{a}^{\star}+k_2\bm{b}^{\star}+k_3\bm{c}^{\star}\\
  &=k'_1\bm{a}^{\star\prime}+k'_2\bm{b}^{\star\prime}+k'_3\bm{c}^{\star\prime},
\end{align} 
the coordinates are mapped with
\begin{equation}
  \begin{pmatrix}
    k_1\\
    k_2\\
    k_3
  \end{pmatrix}
  =
  {M^{-1}}^{T}
  \begin{pmatrix}
    k'_1\\
    k'_2\\
    k'_3
  \end{pmatrix}.
\end{equation} 

\subsection{Band representations: from the group to the subgroup}
For a magnetic space group, a band representation can be specified by the counts of the little group irreducible representations arising at (eight or fewer) maximal high symmetry points in reciprocal space. Given a band representation of a magnetic space group $G$, we seek to the derive the band representation of a subgroup $G'\subset G$.

To this end, it suffices to obtain a decomposition of the irreps of the little groups of $G$ into irreps of the little groups of $G'$, at the maximal $k$-points of $G'$. It is important to note that such a decomposition depends on the \emph{concrete} subgroup $G'$, and not only on its multiplication table. That is, two subgroups $G'_1,G'_2\subset G$ sharing the same magnetic space group number but having different \emph{standard} lattice translation vectors (as specified by $\mathsf{M}$ in Eq.~\ref{subgroupToGroupTranslations}) will generally have distinct decomposition tables. For example, the symmetry of a solid initially described by the magnetic space group symmetry $Pmmm~(47.249)$ will be to reduced to a subgroup labelled $P2/m~(10.42)$ upon the application of an external magnetic field along one of the \emph{inequivalent} directions $[100]$, $[010]$ and $[001]$. Despite having the same magnetic space group number, the three subgroups are in fact distinct groups, each with a different high symmetry axis and a distinct decomposition of the original group irreps.

\subsection{Algorithm}
\begin{itemize}
  \item Input: Group $G$, Wyckoff Site $W$.
  \item Algorithm:
    \begin{enumerate}
      \item for each maximal $\bm{k}$-vector of $G$, find the multiset $\mathtt{irreps}({\bm{k}})=\{\rho_{i,\bm{k}}\}$ consisting of the magnon band structure irreps at $\bm{k}$.
      \item for each $G'\subset G$:
        \begin{enumerate}
          \item find a band structure basis $\{\bm{b}\}$ and the symmetry indicators $\{\mathrm{si}\}$ of $G'$.
          \item find $d(\bm{k}'):\bm{k}'\mapsto{\{(\rho_{i,\bm{k}},{\{\rho'_{j,\bm{k}'}\}})\}}$, a map from a subgroup maximal vector $\bm{k}'$ to a set of superirreps at the corresponding supergroup maximal vector $\bm{k}$ and their decomposition at $\bm{k}'$.
          \item for each $s\in S(\mathtt{irreps}(\bm{k}_{1}))\times S(\mathtt{irreps}(\bm{k}_{2}))\times\cdots$, where $S(\mathtt{irreps}(\bm{k}))$ is the set of all permutation sequences of magnon irreps at $\bm{k}$, do the following:
            \begin{enumerate}
              % \item Split $s$ into the sequence $\sigma_1,\sigma_2,\cdots$, where each $\sigma_i$ is composed of connected bands satisfying the compatibility relations
              \item for each subgroup-maximal $\bm{k}'$
            \end{enumerate}
        \end{enumerate}
    \end{enumerate}
\end{itemize}


\subsection{Mapping $k$-labels from the subgroup to original group}
A vector in reciprocal space can be written as
\begin{equation}
  \bm{k} =
  \begin{pmatrix}
    k^x\\
    k^y\\
    k^z
  \end{pmatrix}
  =
  \begin{pmatrix}
    \bm{a}^\star&
    \bm{b}^\star&
    \bm{c}^\star
  \end{pmatrix}
  \begin{pmatrix}
    k_1\\
    k_2\\
    k_3
  \end{pmatrix}
\end{equation} 
where $(k^x,k^y,k^z)$ are Cartesian coordinates, and $(k_1,k_2,k_3)$ are coordinates in the basis of conventional reciprocal translations $(\bm{a}^\star,\bm{b}^\star,\bm{c}^\star)$. The latter are defined in terms of the conventional direct-space lattice translations $(\bm{a},\bm{b},\bm{c})$ as
\begin{equation}
  \begin{pmatrix}
\bm{a}^\star&\bm{b}^\star&\bm{c}^\star
  \end{pmatrix}
={{\begin{pmatrix}
    \bm{a}&\bm{b}&\bm{c}
    \end{pmatrix}}^{-1} }^{T}.
\end{equation}

Under a rotation, the conventional coordinates $(k_1,k_2,k_3)$ generally transform by a non-orthogonal matrix, since $\bm{a}^\star,\bm{b}^\star$ and $\bm{c}^\star$ are not orthonormal, in general.


Let $G$ and $G'$ be two magnetic space groups where $G'\subset G$, and denote by $(\bm{a},\bm{b},\bm{c})$ and $(\bm{a}',\bm{b}',\bm{c}')$ the corresponding conventional translations. The relation between the two sets of translation vectors is given by the 3\texttimes3 matrix $M$ defined by
\begin{equation}
  \begin{pmatrix}
    \bm{a}'&\bm{b}'&\bm{c}'
  \end{pmatrix}
  =
  \begin{pmatrix}
    \bm{a}&\bm{b}&\bm{c}
  \end{pmatrix}
  M.
\end{equation} 
\begin{equation}
  \left(
  \begin{array}{@{}c|c@{}}
    \begin{matrix}
      \phantom{0}&& \\
      \bm{a}'&\bm{b}'&\bm{c}' \\
      \phantom{0}&& 
    \end{matrix}&\bm{o}'\\
    \hline
    \begin{matrix}
      0&0&0
    \end{matrix}&1
  \end{array}
  \right)
  =
  \left(
  \begin{array}{@{}c|c@{}}
    \begin{matrix}
      \phantom{0}&& \\
      \bm{a}&\bm{b}&\bm{c} \\
      \phantom{0}&& 
    \end{matrix}&\bm{o}\\
    \hline
    \begin{matrix}
      0&0&0
    \end{matrix}&1
  \end{array}
  \right)
  \left(
  \begin{array}{@{}c|c@{}}
    \begin{matrix}
      \phantom{0}&& \\
      &M& \\
      \phantom{0}&& 
    \end{matrix}&\bm{\Delta}\\
    \hline
    \begin{matrix}
      0&0&0
    \end{matrix}&1
  \end{array}
  \right)
\end{equation} 
The reciprocal vectors of the original group and the subgroup are therefore related by
\begin{equation}
  \begin{pmatrix}
    \bm{a}^{\star\prime}&\bm{b}^{\star\prime}&\bm{c}^{\star\prime}
  \end{pmatrix}
  =
  \begin{pmatrix}
    \bm{a}^\star&\bm{b}^\star&\bm{c}^\star
  \end{pmatrix}
  {M^{-1}}^{T}.
\end{equation} 

Using a conventional setting with translation vectors $(\bm{a},\bm{b},\bm{c})$ and origin $\bm{o}$, a site $\bm{r}=(r^x,r^y,r^z)$  reads
\begin{equation}
  \bm{r}=r_1 \bm{a}+r_2 \bm{b}+r_3 \bm{c}+\bm{o},
\end{equation} 
or, equivalently,
\begin{equation}
  \begin{pmatrix}
    \bm{r}\\
    1
  \end{pmatrix}
  =
  \begin{pmatrix}
    r^x\\
    r^y\\
    r^z\\
    1
  \end{pmatrix}
  =
  \left(
  \begin{array}{@{}ccc|c@{}}
    \phantom{0}&&&\\
    \bm{a}&
    \bm{b}&
    \bm{c}&
    \bm{o}\\
    \phantom{0}&&&\\
    \hline
    0&0&0&1
  \end{array}
  \right)
  \begin{pmatrix}
    r_1\\
    r_2\\
    r_3\\
    1
  \end{pmatrix},
\end{equation} 
a notation which can be more convenient for our purposes. For a crystal symmetry $g\in G$, the coordinates transform by
\begin{equation}
  g:\quad
  \begin{pmatrix}
    r_1\\
    r_2\\
    r_3\\
    1
  \end{pmatrix}
  \mapsto
  \left(
  \begin{array}{@{}c|c@{}}
    \begin{matrix}
      \phantom{0} \\
      R_g \\
      \phantom{0} 
    \end{matrix}&\bm{\delta}_g\\
    \hline
    \begin{matrix}
      0&0&0
    \end{matrix}&1
  \end{array}
  \right)
  \begin{pmatrix}
    r_1\\
    r_2\\
    r_3\\
    1
  \end{pmatrix}
\end{equation}
On the other hand, the subgroup coordinates $(r_1',r_2',r_3')$ transform by
\begin{equation}
  g:\quad
  \begin{pmatrix}
    r_1'\\
    r_2'\\
    r_3'\\
    1
  \end{pmatrix}
  \mapsto
  \left(
  \begin{array}{@{}c|c@{}}
    \begin{matrix}
      \phantom{0} \\
      R_g' \\
      \phantom{0} 
    \end{matrix}&\bm{\delta}'_g\\
    \hline
    \begin{matrix}
      0&0&0
    \end{matrix}&1
  \end{array}
  \right)
  \begin{pmatrix}
    r_1'\\
    r_2'\\
    r_3'\\
    1
  \end{pmatrix}
\end{equation}
where
\begin{equation}
  R'_g=M^{-1}R_g M.
\end{equation} 
Thus, the action of a symmetry element $g$ on $\bm{k}'$ with components written in the conventional basis of the subgroup reads
\begin{equation}
  g:\bm{k}'\mapsto{{(M^{-1}R_{g}M)}^{-1}}^{T}\bm{k}'
\end{equation} 
where $R_{g}$ is \emph{the} 3\texttimes3 matrix implementing the action of $g$ in \emph{direct space} in the conventional basis of the \emph{original} group.

{\bf{\color{red}N\color{orange}o\color{green}t\color{blue}e\color{violet}:}}
\begin{equation}
  {\left(
  \begin{array}{@{}c|c@{}}
    \begin{matrix}
      \phantom{0} \\
      R \\
      \phantom{0} 
    \end{matrix}&\bm{\delta}\\
    \hline
    \begin{matrix}
      0&0&0
    \end{matrix}&1
  \end{array}
  \right)}^{-1}
  =
  \left(
  \begin{array}{@{}c|c@{}}
    \begin{matrix}
      \phantom{0} \\
      R^{-1} \\
      \phantom{0} 
    \end{matrix}&-R^{-1}\bm{\delta}\\
    \hline
    \begin{matrix}
      0&0&0
    \end{matrix}&1
  \end{array}
  \right)
\end{equation}

\subsection{Summary}

\begin{equation}
  \begin{pmatrix}
    \bm{a}' & \bm{b}' & \bm{c}'
  \end{pmatrix}
  =
  \begin{pmatrix}
    \bm{a} & \bm{b} & \bm{c}
  \end{pmatrix}
  M
\end{equation} 

\begin{equation}
  \begin{pmatrix}
    r_1\\
    r_2\\
    r_3
  \end{pmatrix}
  = M
  \begin{pmatrix}
    r'_1\\
    r'_2\\
    r'_3
  \end{pmatrix}
\end{equation} 
\begin{equation}
  \begin{pmatrix}
    \bm{a}^{\star\prime} & \bm{b}^{\star\prime} & \bm{c}^{\star\prime}
  \end{pmatrix}
  =
  \begin{pmatrix}
    \bm{a}^{\star} & \bm{b}^{\star} & \bm{c}^{\star}
  \end{pmatrix}
  {M^{-1}}^{T}
\end{equation} 

\begin{equation}
\begin{equation}
  \begin{pmatrix}
    k_1\\
    k_2\\
    k_3
  \end{pmatrix}
  = {M^{-1}}^{T}
  \begin{pmatrix}
    k'_1\\
    k'_2\\
    k'_3
  \end{pmatrix}
\end{equation} 

\begin{equation}
  g:\quad
  \begin{pmatrix}
    r_1\\
    r_2\\
    r_3
  \end{pmatrix}
  \mapsto R_{g}
  \begin{pmatrix}
    r_1\\
    r_2\\
    r_3
  \end{pmatrix}
\end{equation} 

\begin{equation}
  g:\quad
  \begin{pmatrix}
    r'_1\\
    r'_2\\
    r'_3
  \end{pmatrix}
  \mapsto M^{-1}R_{g}M
  \begin{pmatrix}
    r'_1\\
    r'_2\\
    r'_3
  \end{pmatrix}
\end{equation} 

\begin{equation}
  g:\quad
  \begin{pmatrix}
    k_1\\
    k_2\\
    k_3
  \end{pmatrix}
  \mapsto {{(R_{g})}^{-1}}^{T}
  \begin{pmatrix}
    k_1\\
    k_2\\
    k_3
  \end{pmatrix}
\end{equation} 

\begin{equation}
  g:\quad
  \begin{pmatrix}
    k'_1\\
    k'_2\\
    k'_3
  \end{pmatrix}
  \mapsto {{(M^{-1}R_{g}M)}^{-1}}^{T}
  \begin{pmatrix}
    k'_1\\
    k'_2\\
    k'_3
  \end{pmatrix}
\end{equation} 

\subsection{Magnetic Space Group}
\subsection{Example: TbFeO$_3$}
\subsubsection{No perturbation}
Without any perturbation, the magnetic space group is $Pn'ma'~(62.448)$ and the Wyckoff site is $4b$.
\subsubsection{$\vec{B}\parallel[001]$}
The group is reduced to $P2_1'/c'~(14.79)$ with real space translations
\begin{equation}
  \begin{pmatrix}
    \pvec{a}'\\\pvec{b}'\\\pvec{c}'
  \end{pmatrix}
  =\begin{pmatrix}
    &&1\\
    1&&\\
    &1&-1
  \end{pmatrix}
  \begin{pmatrix}
    \vec{a}\\\vec{b}\\\vec{c}
  \end{pmatrix}
\end{equation} 
and reciprocal space translations
\begin{equation}
  \begin{pmatrix}
    \pvec{a}^{\star'}\\\pvec{b}^{\star'}\\\pvec{c}^{\star'}
  \end{pmatrix}
  =\begin{pmatrix}
    &&1\\
    1&&\\
    &1&-1
  \end{pmatrix}
  \begin{pmatrix}
    \pvec{a}^\star\\\pvec{b}^\star\\\pvec{c}^\star
  \end{pmatrix}
\end{equation} 
\end{document}
